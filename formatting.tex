\usepackage{titlesec}
\usepackage{parskip}
\usepackage{xcolor}
\usepackage{multicol}
\usepackage{array}
\usepackage{longtable}
\usepackage{tabularx}
\usepackage{enumitem}

% ~~~~~~~~~~~~~~~~
% PAGE SETUP 
% ~~~~~~~~~~~~~~~~
\usepackage{geometry}

\geometry{
	paper=a4paper,
	top=1.75cm,
	bottom=1.75cm,
	left=2cm,
	right=2cm,
	headheight=0.75cm,
	footskip=1cm,
	headsep=0.5cm,
	%showframe, % Uncomment to show how the type block is set on the page
}
\pagestyle{empty}

\setlength{\LTpre}{0pt}
\setlength{\LTpost}{0pt}
\setlength{\tabcolsep}{0pt}

% ~~~~~~~~~~~~~~~~
% FONTS & ICONS
% ~~~~~~~~~~~~~~~~
\usepackage{fontawesome5}
\usepackage{fontspec}
\setmainfont{Raleway}[Weight=Light]

\newcommand{\icon}[3][1em]{
	\colorbox{black}{\makebox[#1]{\textcolor{white}{\large\csname fa#2\endcsname}}}
	\hspace{0.2cm}
	\textcolor{black}{#3}
}

% ~~~~~~~~~~~~~~~~
% CUSTOM COMMANDS
% ~~~~~~~~~~~~~~~~

\newcommand{\cvsect}[1]{% The only parameter is the section text
	\vspace{\baselineskip} % Whitespace before the section title
	\colorbox{black}{\textcolor{white}{\MakeUppercase{\textbf{#1}}}}\\% Section title
}

\newcommand{\cvbreak}{
	\vspace{\baselineskip}
	\noindent\rule{\textwidth}{0.6pt}
}

\newcommand{\cvexperience}[4]{%
	\begin{tabularx}{\textwidth}{@{}X r@{}}
		\textbf{#1} — #2 & \footnotesize #3 \\
	\end{tabularx}
	\vspace{-0.8em}
	#4
	\vspace{0.6em}
}

\newcommand{\cveducation}[3]{%
	\begin{tabularx}{\textwidth}{@{}X r@{}}
	  \textbf{#1} & \footnotesize #2 \\
	\end{tabularx}
	#3
	\vspace{0.6em}
}
